\section{Validators}
A collaborator should only sign a state update, as described in \ref{sec:reach_consensus}, if it is considered to be valid. In the following paragraphs, we describe both the protocol-level validation rules as well as document-specific rules.

\subsection{Protobuf Validation}
A document consists of two parts. The schema fields (Sec. \ref{sec:doc_schema_fields}) depend on the document type (e.g. \textit{Purchase Order}, \textit{Invoice}). The core document fields (Sec. \ref{sec:doc_core_doc_fields}) are part of every document. Protobuf's schema definitions provides significant built-in validation by enforcing types and fields. When exchanging the document via P2P messages, they are marshalled into the protobuf byte format and thus type-enforced. We will skip over these validation rules and assume that a user would always reject a document that is an invalid protobuf message.

\subsection{Schema Validation}
A validator defines a set of transition rules for a document field or a set of document fields. For example, it would be not allowed to change the order quantity field of a purchase order to a negative value. Formally, a validator $V$ is a function which has as an input the current document $d$ and the next document $d'$.

\begin{eqnarray}
v & = & \mathsf{V}(d, d') \\
 v &\in & \{0,1\}
\end{eqnarray}
The output of a validator function $V$ is either $1$ for accepted or $0$ for denied. If the field transitions are described in the validator $V$ the validator returns $1$.
The validator $V$ must also define case $d = \emptyset$ for a given first version of a document.

Every document of a specific schema $\Upsilon$  has a set of different document validators for describing a document schema like invoice or purchaseorder.
\begin{equation}
     \Upsilon = (V_0,...,V_n) 
\end{equation}
We define a $\mathsf{validateSchema}$ which evaluates all Validators $V$ of a schema $\Upsilon$

\begin{equation}
\begin{split}
   \mathsf{validateSchema}(d_S,d'_S,\Upsilon) =\prod_{i=0}^{n} V_i(d, d')  \qquad \forall V \in \Upsilon
    \end{split}
\end{equation}
All validators need to accept the proposed document $d'$. Otherwise, the function will return $0$ for denied.
\begin{equation}
  \mathsf{validateSchema}(d_S,d'_S,\Upsilon) = 1
\end{equation}
 In the special case of the first document $d_{[0]}$, schema validation must occur.
\begin{equation}
  \mathsf{validateSchema}(\emptyset, d'_S,\Upsilon) = 1 \quad \text{for} \quad  d' = d_{[0]}
\end{equation}

\subsection{Validate Rules}
A document has specific rules which define the read and write permissions field by field. A rule defines a set of roles which are allowed to write, read or should sign a document. (See Section:  \ref{sec:roles_rules}) A proposed new document $d'$ is not allowed to violate the rules. Therefore for each changed field in the document $d'$ the the rules need be evaluated.

\begin{equation}
 \begin{split}
  r = \mathsf{validateRules}(d,d',d'_{\mathtt{author}},d'_{\mathtt{rules}}) \\
  r \in \{1;0\}
 \end{split}
\end{equation}

The rules together with specific field are also used to implement document acceptance on the business layer. Let's say a invoice document has two defined roles, one for the supplier and one for the buyer. The invoice could have a specific field \textit{buyer-accepts-invoice} together with a specific rule which allows only members of the role buyer to modify it. In case the supplier wants to set the field to true the buyer would reject the signature. 

\subsection{Validate Function}
A user only signs the signing root of a document  only if all validators in the schema are accepted and if no document rule is violated.
\begin{equation}
 \mathsf{validate}(d,d') =\mathsf{validateSchema}(d,d',\Upsilon) * \mathsf{validateRules}(d,d',d'_{\mathtt{author}},d'_{\mathtt{rules}}) \\
 \end{equation}
    \begin{equation}
 \mathsf{validate}(d,d') = 1
 \end{equation}

\subsubsection{Validation}
If the user has both the previous and the current version of the document, it will sign off on the state update. A document signature of a collaborator on both $d$ and $d\prime$ means the collaborator validated both the format of $d$ and $d\prime$. It also indicates that the user has validated that no validation rules were violated transitioning from $d$ to $d\prime$.


 
 
 
