\section{User Mintable Non-Fungible Tokens}
\label{sec:nft}
To allow for the integration of Centrifuge documents into other applications built on Ethereum, Centrifuge implements a smart contract library for minting non-fungible tokens (NFTs as specificed by ERC721). NFTs are a widely adopted standard to create objects that have trackable and unique ownership. Centrifuge extends this standard with methods that allow users to mint these NFTs based on the private off-chain data in a document. When invoking the mint method on an $\mathsf{NFTRegistry}$, a set of fields of the document have to be revealed as merkle proofs, using the precise-proofs library. This information can then be stored in the token registry. We call this data metadata. Semantically, an NFT can have any kind of meaning for the application using it. The protocol itself does not impose any meaning to the tokens but merely provides a verifiable way to mint tokens for specific document states. The contract library provides the helpers to ensure that for one document only one token is minted in an NFT registry, that certain access rights are granted, and that the document is of a valid structure. Centrifuge, however, does define and deploy a number of registries that have a semantic meaning, most notably the Payment Obligation NFT. %TODO Reference 

\subsection{Non-Fungible Token Contract}
An NFT is a unique token that is always owned by one entity (an Ethereum account or contract). A token registry contract is a smart contract that manages a set of tokens.
\begin{description}
\item{\textbf{Registry Contract}} is a smart contract implementing the logic to create and transfer tokens. We define it as $\mathsf{NFTRegistry}$
\item{\textbf{Registry Contract Address}} the Ethereum address of the  $\mathsf{NFTRegistry}$ 
\item{\textbf{Token Identifier}} a $\mathbb{B}_{32}$ value defines the unique identifier of the token in the given registry contract, formally $T_{\mathtt{id}}$
\item{\textbf{Owner}} a $\mathbb{B}_{20}$ value of the Ethereum account or contract that is the owner of the token, formally $T_{\mathtt{owner}}$
\item{\textbf{Anchor}} a $\mathbb{B}_{32}$ value of the document identifier that the token is minted for, formally $T_{\mathtt{anchor}}$  %TODO: Ref to anchor
\item{\textbf{Metadata}} additional data that is supplied when the token is minted, formally $T_{\mathtt{meta-data}}$
\item{\textbf{Constraints}} defines an array of constraints for specific document fields. It  defines required values for the concrete document fields. For example, the an invoice status field to be \textit{unpaid} formally $T_{\mathtt{constraints}}$
\end{description}
\subsection{Minting a Token}
NFTs are created by calling the \texttt{mint} method of an NFT contract. The \texttt{mint} method requires the submission of a set of Merkle proofs for specific fields of the document for which the token has been minted. E.g. proofs for the invoice due date, amount, and currency in order to mint a token representing a payment obligation NFT. The Merkle proofs are validated by the NFT contract and checked against the document root provided by the Centrifuge.
\textit{AnchorRepository}. Any user who can produce valid proofs can mint the NFT. Together with the Merkle proof a second contract call is sometimes required. For example to proof a signature the Merkle proof verifies that a signature is part of a the document and a second call to an identity contract call would verify the signature itself.
\\The mint method can be defined as:\\
\begin{equation}
    \textsf{mint}(T_{\texttt{owner}}, T_{\texttt{id}},T_{\mathtt{meta-data}},d_{\mathtt{current}},d_{\mathtt{next}}, (P_{0},\dots, P_{n})) \xrightarrow[]{\text{ETH}} \mathsf{Identity} \xrightarrow[]{\text{ETH}}  \mathsf{NFTRegistry}
    %M_{\texttt{proof}}(R_{\texttt{doc-root}})
\end{equation}\\
A node doesn't call the mint method of the NFT Registry directly. The proxy functionality of the identity contract is used together with an action key $k_\mathtt{action}$ \textsf{mint} stored inside the node.\\\\
The mint method requires a set of proofs $(P_{[0]},...,P_{[n]})$ to perform the merkle proofs.
\begin{equation}
\begin{split}
P = (P_{\mathtt{field}},P_{\mathtt{sibling-hashes}}) \qquad
P_{\mathtt{field}} \in F \quad \And \quad P_{\mathtt{sibling-hashes}} \in \mathbb{B}_{32}^n 
\end{split}
\end{equation}

\subsubsection{Getting the Anchors}
The mint method needs the anchor for $d_{\mathtt{current}}$ to perform the merkle proofs based on the document root.
\begin{equation}
\begin{split}
{\scriptscriptstyle \mathsf{NFTRegistry}}  {\scriptscriptstyle &\xleftarrow[]{\text{ETH}}}  {\scriptscriptstyle \mathsf{AnchorRepository}} \\
{A_{\texttt{anchor-current}} &\xleftarrow[]{\text{ETH}} \textsf{getCommit}(d_{\mathtt{current}})}
\end{split}
\end{equation}
\textsf{NFTRegistry} calls the \textsf{AnchorRepository} to receive the document root (anchor) of the document. 

\subsubsection{Merkle Proof}
A Merkle proof can be seen as a function which proves if a provided field $P_\mathtt{field}$ is part of document. By calling the NFT mint method the value of these fields is revealed to the public. 
\begin{equation}
\begin{split}
v = \mathcal{M}_{\texttt{proof}}(A_{\texttt{anchor-current}},P_\mathtt{field},P_\mathtt{sibling-hashes}) \\
v \in \{0;1\}
\end{split}
\end{equation}
The $P_\mathtt{field}$  could be any field of the core document $d_C$ or the schema field $d_S$. The required $P_\mathtt{sibling-hashes}$ can be generated with the precise proofs library inside the node and contains the siblings of the field in each layer of the merkle tree.  
\subsubsection{Individual Field Merkle Proof}
Every NFT registry defines a set of mandatory fields on the document which need to be revealed. The mandatory fields depend on the purpose of the NFT and are defined by the initialization of the smart contract.
\begin{equation}
\begin{split}
\mathcal{M}_{\texttt{proof}}(A_{\texttt{anchor-current}} ,P_{[i]\mathtt{field}},P_{[i]\mathtt{sibling-hashes}}) = 1
\end{split}
\end{equation}
Every mandatory field requires a proof to verify that it is part of the document. Some mandatory fields could require the $P_{[i]\mathtt{field.value}}$ to have a concrete value.
\begin{equation}
    P_{[i]\mathtt{field.value}} = T_{\mathtt{constraints}[\mathtt{field.name}]}
\end{equation}
The required values are defined in the contract itself and are formally noted as $T_{\mathtt{constraints}}$. An example could be the status field of an invoice needs to be \textit{unpaid} for a specific NFT registry. It is possible to implement more complex conditions in a registry contract. Like the value of a fields needs to equal one out of $n$ pre defined values. 
\subsubsection{Latest Document}
A document can have multiple versions. It should be only possible to mint an NFT out of the latest document version. The $d_{\mathtt{next}}$ defines the anchor id of the next version. The mint method receives $d_{\mathtt{next}}$ as a parameter and can fetch the $A_{\texttt{anchor-next}}$ from the anchor registry.
\begin{equation}
\begin{split}
\mathsf{NFTRegistry} &\xleftarrow[]{\text{ETH}}  \mathsf{AnchorRepository} \\
A_{\texttt{anchor-next}} &\xleftarrow[]{\text{ETH}} \textsf{getCommit}(d_{\mathtt{next}})
\end{split}
\end{equation}
If $A_{\mathtt{anchor-next}}$ is not existing in the anchor registry and the $d_{\mathtt{next}}$ is part of document root.
\begin{equation}
\begin{split}
A_{\texttt{anchor-next}}  = \emptyset \quad \land \quad
\mathcal{M}_{\texttt{proof}}(A_{\texttt{anchor-current}} ,P_{[\mathtt{next}]\mathtt{field}}, &P_{[\mathtt{next}]\mathtt{sibling-hashes}})  = 1
\end{split}
\end{equation}
The $A_{\mathtt{anchor-current}}$ is the document root and $d_\mathtt{current}$ is the id of the latest document version.
\subsubsection{Verify an identity}
A NFT registry should be called by an identity contract. It makes it easier to verify signatures later on. Only identity contracts created by the Centrifuge Identity factory are accepted. Every provided collaborator DID can be verified by checking if the factory contract created the collaborator's identity contract. First, it needs to be verified that the collaborators $DID$ is part of the document.
\begin{equation}
\mathtt{msg.sender} = I_{\mathtt{DID}}
\end{equation}
For the document collaborator who wants to mint the document the Ethereum \texttt{msg.sender} is equal to their $DID$. In a second step the identity factory is called to verify if it an identity contract.
\begin{equation}
\begin{split}
\mathcal{M}_{\texttt{proof}}(A_{\texttt{anchor-current}} ,P_{[\mathtt{collaborator.did}]\mathtt{field}}, &P_{[\mathtt{collaborator.did}]\mathtt{sibling-hashes}})  = 1 \\
\mathsf{NFTRegistry} &\xleftarrow[]{\text{ETH}}  \mathsf{IdentityFactory} \\
v &\xleftarrow[]{\text{ETH}} \textsf{createdIdentity}(\mathtt{P_{[\mathtt{collaborator.did}]}}) \\
v & = 1 \\
\end{split}
\end{equation}

\subsubsection{Verify a Signature}
Providing a signature of a collaborator as part of a NFT minting ensures that the collaborator agreed on the document state. Agreement means a document rule $d_rules$ or a schema validation was not violated. For verifying a signature of the document signing root $R_\mathtt{signing}$ five steps are required.
\textbf{Signature part of document} \\
First a merkle proof verifies that a provided signature is part of the document.
\begin{equation}
\begin{split}
\mathcal{M}_{\texttt{proof}}(A_{\texttt{anchor-current}} ,P_{[\mathtt{signature}]\mathtt{field}},P_{[\mathtt{signature}]\mathtt{sibling-hashes}}) = 1 \\
\end{split}
\end{equation}
\textbf{Collaborator part of the document} \\
In a second merkle proof the DID of the collaborator who signed the document needs to be part of the document collaborators.
\begin{equation}
\begin{split}
\mathcal{M}_{\texttt{proof}}(A_{\texttt{anchor-current}} ,P_{[\mathtt{collaborator.did}]\mathtt{field}},P_{[\mathtt{collaborator.did}]\mathtt{sibling-hashes}}) = 1 \\
\end{split}
\end{equation}
\textbf{Signing root part of the document} \\
A third merkle proof verifies that the siging root $R_\mathtt{signing}$ is part of the document.
\begin{equation}
\begin{split}
\mathcal{M}_{\texttt{proof}}(A_{\texttt{anchor-current}} ,P_{[\mathtt{signing-root}]\mathtt{field}},P_{[\mathtt{signing-root}]\mathtt{sibling-hashes}}) = 1 \\
\end{split}
\end{equation}
\textbf{Collaborator identity verification} \\
The DID of the collaborator needs to be an ERC725 identity contract created by the Identity factory.
\begin{equation}
\begin{split}
\mathsf{NFTRegistry} &\xleftarrow[]{\text{ETH}}  \mathsf{IdentityFactory} \\
v &\xleftarrow[]{\text{ETH}} \textsf{createdIdentity}(\mathtt{msg.sender}) \\
v & = 1
\end{split}
\end{equation}
\textbf{Verify the signature of the Collaborator} \\
If the DID of the collaborator is a ERC725 identity contract. The NFTRegistry can call the signature verification method on it.
\begin{equation}
\begin{split}
\mathsf{NFTRegistry} &\xleftarrow[]{\text{ETH}}  \mathsf{Identity} \\
K_{\texttt{purpose}} & = \mathtt{sha256("CENTRIFUGE@SIGNING"))} \\
v &\xleftarrow[]{\text{ETH}} \textsf{isSignedWithPurpose}(P_{[\mathtt{signing-root}]\mathtt{field}},P_{[\mathtt{signature}]\mathtt{field}},K_{\texttt{purpose}}))\\
v & = 1
\end{split}
\end{equation}
\\
If the signature was signed with the purpose key for signing the signature is accepted by the mint method.

\subsubsection{Uniqueness Enforcement}
A document can have multiple different NFT's for different purposes. Nevertheless it should be not possible to mint the exact same NFT twice for the same document in the same registry. For example an NFT which could represent a payment obligation of an invoice should be only allowed to be minted once. For enforce the requirement the NFT id $T_{id}$ together with the NFT registry address needs to be part of the document. 
\begin{equation}
\begin{split}
N & = (N_{\mathtt{registry-id}}, N_{\mathtt{token-id}}) \\
N_{\mathtt{registry-id}} \in \mathbb{B}_{32}  \quad &\land \quad N_{\mathtt{token-id}} \in \mathbb{B}_{32}  \\
d_{\mathtt{nfts}}& = (N_{[0]},...,N_{[n]})
\end{split}
\end{equation}
A collaborator who wants to mint an NFT first needs to create a new document version $d'$.
\begin{equation}
\begin{split}
N'_{\mathtt{token-id}} & = \texttt{RAND(32)}\\
N'_{\mathtt{registry-id}} & = T_{\mathtt{registry-address}} \quad \| \quad \mathbb{B}_{12} \\
N'  & = (N'_{\mathtt{registry-id}},N'_{\mathtt{token-id}}) \\
d' & = d \\
d'_{\mathtt{nfts}} & = \texttt{append}(d_{\mathtt{nfts}},N')
\end{split}
\end{equation}
The new document version $d'$ needs to include the token id $T_{\mathtt{id}}$ which can be randomly chosen by
the user and the registry address itself. In Protobuf definition of the document $d_{\mathtt{nfts}}$ is implemented as a map where the key is 32 bytes including the 20 bytes of the Ethereum registry address and 12 bytes of zeros in the end. The user needs to request the signatures of the collaborators and commit the new document version as described in a previous chapter. (See chapter \ref{sec:consensus})
\begin{equation}
\begin{split}
\mathcal{M}_{\texttt{proof}}(A_{\texttt{anchor-current}} ,P_{[\mathtt{nfts]{[registry-address]}}\mathtt{field}},P_{[\mathtt{nfts]{[registry-address]}}\mathtt{sibling-hashes}}) &= 1 \\
P_{[\mathtt{nfts]{[registry-address]}}\mathtt{field.value}} &= T_{\mathtt{id}}
\end{split}
\end{equation}
Inside the mint method a merkle proofs verifies that the provided $T_{id}$ is part of the document. A collaborator would not sign a new document version $d'$ if the current one already includes a token id for the same NFT registry. Without the signatures of the collaborators it is not possible to mint a token for the same registry twice.


\subsection{Token Methods}
The ERC721 standard defines the interface for a NFT registry.
\subsubsection{Transfer}
The token smart contract has a function to get the owner of a token by the token identifier. This method is compliant with the ERC721 standard

\begin{equation}
 T_{\mathtt{owner}} \xleftarrow[]{\text{ETH}} \mathsf{ownerOf}(T_{\texttt{id}})
\end{equation}
A token can be transferred by the current owner to a different owner:

\begin{equation}
\mathsf{transferTo}(T_{\texttt{id}},T_{\texttt{owner}},T_{\texttt{new-owner}}) \xrightarrow[]{\text{ETH}}  \mathsf{NFTRegistry}
\end{equation}
\subsubsection{Metadata}
Metadata 
 

\subsection{Document Access}
Over the life time of an NFT it can be transfered to different users and companies. A person or a user who owns an NFT has the option to request read access to the document which belongs to the NFT. The token owner needs to have a node and a DID. It is not required that the owner is in the list of document collaborators $d_{\mathtt{collaborators}}$.
\begin{equation}
    \begin{split}
            (\mathtt{DID_{sender}},T_{\mathtt{id}},T_{\mathtt{registry-address}},d_{\mathtt{id}})& \xleftarrow[]{\text{P2P}}  \mathsf{receiveDocumentRequest}(\mathtt{DID_{sender}}) \\
            T_{\mathtt{owner}}  & \xleftarrow[]{\text{ETH}} \mathsf{ownerOf}(T_{\texttt{id}}) \\
               d    &= \mathsf{loadLatestVersion}({d}_{\mathtt{id}}) \\
    \end{split}
\end{equation}
If the following condition is true the $\mathtt{DID_{sender}}$ is a NFT owner of the document.
\begin{equation}
    \begin{split}
     d_{\mathtt{nfts[T_{registry-address}]}} = T_{\mathtt{id}} \quad \land \quad T_{\mathtt{owner}} = \mathtt{DID_{sender}}
    \end{split}
\end{equation}
The provided token id $T_{\mathtt{id}}$ needs to be part of the document and the NFT owner in the registry needs to be the $\mathtt{DID_{sender}}$. If both requirements are fullfilled the document will be send.
\begin{equation}
      \mathsf{send}(d) \xrightarrow[]{\text{P2P}} \mathtt{DID_{sender}}
\end{equation}


