\section{Document State Consensus}
\label{sec:consensus}
In \ref{sec:doc_state_comm}, we introduced how state is committed to and in \ref{sec:particpiants}, we defined the attributes and behaviors of individual actors in the system. Building upon this, we now introduce our mechanism for enabling the reaching of consensus among different parties.
\subsection{Consensus Trade-offs}
For the initial protocol version we have decided to avoid a fully BFT consensus scheme. The protocol does not guarantee data availability for every participant. Instead, the focus is on preserving privacy in the optimistic case and requiring any consensus failures to be settled out of band. 
\subsection{Reaching consensus on the document state}\label{sec:reach_consensus}
When a collaborator wants to update the state and commit to a new version, the document changes are first sent to all collaborators on a document in order to collect the signatures by the collaborators. The collaborators sign the signing root, $R_{signing}$.

In the following section, the pre-commit and commit is described. A user does not need to use a pre-commit and can commit a document state immediately. However this has drawbacks if there are byzantine nodes.

\subsubsection{Consensus Protocol}
Each collaborator, an identity $I$, is found in the P2P network and the new state is transmitted with a request to provide a signature. We define the user who initiates the state update as $\iota$ and collaborators $c_n$.
\begin{eqnarray}
\iota,c \in I\\
     d_{\mathtt{collaborators}} = (\iota, c_{[0]}, ..., c_{[n]})
\end{eqnarray}

We begin with a document $d$ that will be updated from $d_{[n]} \to d_{[n+1]}$ or in shortened form from $d \to d'$. 

\begin{eqnarray}
    d' & = & d_{[n+1]}
\end{eqnarray}

When proposing a new state $d'$, the user $\iota$ will set the coredocument field author to its $\mathtt{DID}$.

\begin{equation}
    {d'}_{\mathtt{author}} = \iota_{\mathtt{DID}}
\end{equation}

The user $\iota$ first calculates the signing root, $R_{\mathtt{signing}}$ and signs it with its own key and adds it to the set of signatures on $d'$.
\begin{eqnarray}
    R_{\mathtt{signing}} & = & \mathsf{calculateSigningRoot}(d') \\
        \mathsf{preCommit}(d'_\mathtt{current},R_{\mathtt{signing}}) &\xrightarrow[]{\text{ETH}} &  \mathsf{AnchorRepository} \\
     s & = & \mathsf{sign}(\iota_{\mathtt{k.sign}}, R_{\mathtt{signing}}) \\
    {d'}_{\mathtt{signatures}[\iota_{\mathtt{DID}]}} & = & s
\end{eqnarray}

The user then pre-commits to the state and gets signatures from all the collaborators. After a successful pre-commit, the node sends the new state $d'$ to all collaborators and requests their signature on the new state.
\begin{equation}
\begin{split}
s & \xleftarrow[]{\text{P2P}} \mathsf{requestSignature}(d', c) \qquad \qquad \qquad \\
d'_{\mathtt{signatures}[c_{\mathtt{DID}}]} & =  \begin{cases}
  s, \quad \quad \text{if} \quad \mathsf{validateSignature}(c, R_{\mathtt{signing}}, s)  = 1 \\      
  \emptyset,  \quad\quad\text{else} \\
\end{cases} \textcolor{white}{x}\forall c \in d_{\mathtt{collaborators}} \backslash \{\iota\}
%TODO: replace whitespace
\end{split}
\end{equation}
If a collaborator refuses to sign, is unreachable or returns an invalid signature, it is not added to $d'_{\mathtt{signatures}}$. 
\begin{comment}
\begin{eqnarray}
    R_{\mathtt{doc-root}} = \mathsf{calculateDocumentRoot}(d') \\
    \mathsf{commit}(d'_{\mathtt{next-img}}, R_{\mathtt{doc-root}},M_{\texttt{proofs}}) \xrightarrow[]{\text{ETH}} \mathsf{AnchorRepository}
\end{eqnarray}
\end{comment}
\subsubsection{Receiving a signature request}
The collaborator $c$ that receives the signature request first validates the update by comparing the proposed updated document to the previous version of the document that it has in store. 
\begin{eqnarray}
   d' & \xleftarrow[]{\text{P2P}} & \mathsf{recieve}(\iota) \\
    d & = &\mathsf{load}(c, {d'}_{\mathtt{prev}}) \\
    \mathsf{validate}(d, d') & = & 1 
\end{eqnarray}
The function \textsf{validate} is explained in detail in the next section.
The collaborator then verifies that there is a valid signature by the sender, $\iota$ on the document and signs it.\\
%\textcolor{gray}{TODO:add more description}.
\begin{eqnarray}
    A_{\mathtt{signing-root}} & \xleftarrow[]{\text{ETH}} & \mathsf{getPreCommit}(d'_{\mathtt{current}}) \\
    R_{\mathtt{signing}} & = & \mathsf{calculateSigningRoot}(d')\\
    A_{\mathtt{signing-root}} = R_{\mathtt{signing}} &\lor & A_{\mathtt{signing-root}} = \emptyset \\
    \mathsf{validateSignature}(\iota, R_{\mathtt{signing}}, {d'}_{\mathtt{signatures}[\iota]}) & = & 1\\
    \mathsf{sign}(c_{\mathtt{k.sign}}, R_{\mathtt{signing}}) & = & s \\
    \mathsf{sendSignature}(s) &\xrightarrow[]{\text{P2P}}& \iota
\end{eqnarray}

\subsubsection{Publishing the commit}

When all signatures have been collected, or before the timeout for the pre-commit expires, the node commits to the state. 

After receiving all the signatures or the before the pre-commit expires the node finally calculates the document root and commits the anchor.  In case of a timeout the node commmits only with the received signatures.
\begin{eqnarray}
R_{\mathtt{doc-root}} & = & \mathsf{calculateDocumentRoot}(d')\\
  \mathsf{commit}(d'_{\mathtt{next-img}}, R_{\mathtt{doc-root}},M_{\texttt{proofs}}) &\xrightarrow[]{\text{ETH}}& \mathsf{AnchorRepository}\\ 
  \end{eqnarray}
The state update is now considered complete and the node $\iota$ sends an update to all collaborators and readers.

\begin{equation}
\begin{split}
    \mathsf{sendDocument}(d') \xrightarrow[]{\text{P2P}}&\quad c \\ \forall c \in   \{ {d'}_{\texttt{collaborators}} \cup {d'}_{\texttt{readers}} \}
    \end{split}
\end{equation}
\subsubsection{Recieving a commit document}
After the user published the commit every collaborator should receive the final version of $d'$ including the collaborators signatures.
\begin{eqnarray}
    d' & \xleftarrow[]{\text{P2P}} & \mathsf{recieve}(\iota) \\
        A_{\mathtt{doc-root}} & \xleftarrow[]{\text{ETH}} & \mathsf{getCommit}(d'_{\mathtt{current}}) \\
    R_{\mathtt{doc-root}} & = & \mathsf{calculateDocumentRoot}(d')  \\
    A_{\mathtt{doc-root}} &= &R_{\mathtt{doc-root}} \\
   \mathsf{validate}(d, d') & = & 1\\
     \mathsf{validateSignatures}(d'_{\mathtt{collaborators}},R_{\mathtt{signing-root}},d'_{\mathtt{signatures}}) & = & 1
\end{eqnarray}
A collaborator can verify the received document $d'$ if actually an anchor has been committed and which other collaborators signed the document.


